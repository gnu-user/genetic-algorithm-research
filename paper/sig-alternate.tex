% This is "sig-alternate.tex" V2.0 May 2012
% This file should be compiled with V2.5 of "sig-alternate.cls" May 2012
%
% This example file demonstrates the use of the 'sig-alternate.cls'
% V2.5 LaTeX2e document class file. It is for those submitting
% articles to ACM Conference Proceedings WHO DO NOT WISH TO
% STRICTLY ADHERE TO THE SIGS (PUBS-BOARD-ENDORSED) STYLE.
% The 'sig-alternate.cls' file will produce a similar-looking,
% albeit, 'tighter' paper resulting in, invariably, fewer pages.
%
% ----------------------------------------------------------------------------------------------------------------
% This .tex file (and associated .cls V2.5) produces:
%       1) The Permission Statement
%       2) The Conference (location) Info information
%       3) The Copyright Line with ACM data
%       4) NO page numbers
%
% as against the acm_proc_article-sp.cls file which
% DOES NOT produce 1) thru' 3) above.
%
% Using 'sig-alternate.cls' you have control, however, from within
% the source .tex file, over both the CopyrightYear
% (defaulted to 200X) and the ACM Copyright Data
% (defaulted to X-XXXXX-XX-X/XX/XX).
% e.g.
% \CopyrightYear{2007} will cause 2007 to appear in the copyright line.
% \crdata{0-12345-67-8/90/12} will cause 0-12345-67-8/90/12 to appear in the copyright line.
%
% ---------------------------------------------------------------------------------------------------------------
% This .tex source is an example which *does* use
% the .bib file (from which the .bbl file % is produced).
% REMEMBER HOWEVER: After having produced the .bbl file,
% and prior to final submission, you *NEED* to 'insert'
% your .bbl file into your source .tex file so as to provide
% ONE 'self-contained' source file.
%
% ================= IF YOU HAVE QUESTIONS =======================
% Questions regarding the SIGS styles, SIGS policies and
% procedures, Conferences etc. should be sent to
% Adrienne Griscti (griscti@acm.org)
%
% Technical questions _only_ to
% Gerald Murray (murray@hq.acm.org)
% ===============================================================
%
% For tracking purposes - this is V2.0 - May 2012

\documentclass{sig-alternate}
\usepackage{epstopdf}
\usepackage{algorithm2e}
%\usepackage{auto-pst-pdf}
\begin{document}
%
% --- Author Metadata here ---
\conferenceinfo{GECCO2014}{2013 Victoria, BC, Canada}
%\CopyrightYear{2007} % Allows default copyright year (20XX) to be over-ridden - IF NEED BE.
%\crdata{0-12345-67-8/90/01}  % Allows default copyright data (0-89791-88-6/97/05) to be over-ridden - IF NEED BE.
% --- End of Author Metadata ---

\title{Enhancing the Performance of Genetic Algorithms in Combinatorial Optimization of Large and Difficult Problems}
\subtitle{Using Variable Adaptive Mutation Rates Controlled by 'Inbreeding'}
%
% You need the command \numberofauthors to handle the 'placement
% and alignment' of the authors beneath the title.
%
% For aesthetic reasons, we recommend 'three authors at a time'
% i.e. three 'name/affiliation blocks' be placed beneath the title.
%
% NOTE: You are NOT restricted in how many 'rows' of
% "name/affiliations" may appear. We just ask that you restrict
% the number of 'columns' to three.
%
% Because of the available 'opening page real-estate'
% we ask you to refrain from putting more than six authors
% (two rows with three columns) beneath the article title.
% More than six makes the first-page appear very cluttered indeed.
%
% Use the \alignauthor commands to handle the names
% and affiliations for an 'aesthetic maximum' of six authors.
% Add names, affiliations, addresses for
% the seventh etc. author(s) as the argument for the
% \additionalauthors command.
% These 'additional authors' will be output/set for you
% without further effort on your part as the last section in
% the body of your article BEFORE References or any Appendices.

\numberofauthors{4} %  in this sample file, there are a *total*
% of EIGHT authors. SIX appear on the 'first-page' (for formatting
% reasons) and the remaining two appear in the \additionalauthors section.
%
\author{
% You can go ahead and credit any number of authors here,
% e.g. one 'row of three' or two rows (consisting of one row of three
% and a second row of one, two or three).
%
% The command \alignauthor (no curly braces needed) should
% precede each author name, affiliation/snail-mail address and
% e-mail address. Additionally, tag each line of
% affiliation/address with \affaddr, and tag the
% e-mail address with \email.
%
% 1st. author
\alignauthor
Daniel Smullen\titlenote{Corresponding author.}\\
       \affaddr{UOIT}\\
       \affaddr{2000 Simcoe St. N}\\
       \affaddr{Oshawa, ON, Canada}\\
       \email{daniel.smullen@uoit.net}
% 2nd. author
\alignauthor
Jonathan Gillett\\
       \affaddr{UOIT}\\
       \affaddr{2000 Simcoe St. N}\\
       \affaddr{Oshawa, ON, Canada}\\
       \email{jonathan.gillett@uoit.net}
% 3rd. author
\alignauthor
Joseph Heron\\
       \affaddr{UOIT}\\
       \affaddr{2000 Simcoe St. N}\\
       \affaddr{Oshawa, ON, Canada}\\
       \email{joseph.heron@uoit.net}
\and  % use '\and' if you need 'another row' of author names
% 4th. author
\alignauthor Shahryar Rahnamayan\\
       \affaddr{UOIT}\\
       \affaddr{2000 Simcoe St. N}\\
       \affaddr{Oshawa, ON, Canada}\\
       \email{shahryar.rahnamayan@uoit.ca}
}
% There's nothing stopping you putting the seventh, eighth, etc.
% author on the opening page (as the 'third row') but we ask,
% for aesthetic reasons that you place these 'additional authors'
% in the \additional authors block, viz.
\additionalauthors{}
% Just remember to make sure that the TOTAL number of authors
% is the number that will appear on the first page PLUS the
% number that will appear in the \additionalauthors section.

\maketitle
\begin{abstract}
Blah blah blah, here is our abstract.
\end{abstract}

\keywords{Genetic Algorithms, Combinatorial Optimization, N Queens Problem, Variable Mutation}




%%%%%%%%%%%%%%%%%%%%%%%%%%%%%%%%%%%%%%%%%%%%%%%%%%%%%%%%%%%%%%%%%
% 
% INTRODUCTION
%
%%%%%%%%%%%%%%%%%%%%%%%%%%%%%%%%%%%%%%%%%%%%%%%%%%%%%%%%%%%%%%%%%
\section{Introduction}


- Model the intro similary to how we did the report for AI

- Summary of the problem explain the challenges of larger N-Queens problems, 
  provide a table showing how quickly the number of solutions grows, there is
  no known method of determining the exact number of solutions.

- Based on the complexity of the problem explain how the overhead of using a bruteforce 
  approach is infeasible for larger N Queens and the justification for stochastic
  methods.

- Emphasize that our application of the problem is finding distinct solutions,
  since this is much harder than just finding solutions (many of which may be
  duplicates in other random stochastic methods).

- The motive for variable mutation
    --> We were not able initially (without using reflection \& rotation like we
        are now) to get all 92 solutions to 8 queens (impossible with fixed mutation to solve
        in a reasonable amount of time)
    
    --> Started looking at biology, were inspired by the fact that nature has
        a natural ordering to prevent inbreeding within organisms (e.g. purebreed
        dogs often have more health problems, easy citation!)
        
    --> Attempted to come up with a way to apply the negative affects of inbreeding 
        that occur in nature to solve our GA problem
        
    --> Our var. mutation solution worked exceptionally well, solved all 92
        solutions nearly instantaneously, motivated us to pursue the implications
        further.


The \textit{proceedings} are the records of a conference.
ACM seeks to give these conference by-products a uniform,
high-quality appearance.  To do this, ACM has some rigid
requirements for the format of the proceedings documents: there
is a specified format (balanced  double columns), a specified
set of fonts (Arial or Helvetica and Times Roman) in
certain specified sizes (for instance, 9 point for body copy),
a specified live area (18 $\times$ 23.5 cm [7" $\times$ 9.25"]) centered on
the page, specified size of margins (1.9 cm [0.75"]) top, (2.54 cm [1"]) bottom
and (1.9 cm [.75"]) left and right; specified column width
(8.45 cm [3.33"]) and gutter size (.83 cm [.33"]).

The good news is, with only a handful of manual
settings\footnote{Two of these, the {\texttt{\char'134 numberofauthors}}
and {\texttt{\char'134 alignauthor}} commands, you have
already used; another, {\texttt{\char'134 balancecolumns}}, will
be used in your very last run of \LaTeX\ to ensure
balanced column heights on the last page.}, the \LaTeX\ document
class file handles all of this for you.

The remainder of this document is concerned with showing, in
the context of an ``actual'' document, the \LaTeX\ commands
specifically available for denoting the structure of a
proceedings paper, rather than with giving rigorous descriptions
or explanations of such commands.

\subsection{Background and Problem Domain}
Typically, the body of a paper is organized
into a hierarchical structure, with numbered or unnumbered
headings for sections, subsections, sub-subsections, and even
smaller sections.  The command \texttt{{\char'134}section} that
precedes this paragraph is part of such a
hierarchy.\footnote{This is the second footnote.  It
starts a series of three footnotes that add nothing
informational, but just give an idea of how footnotes work
and look. It is a wordy one, just so you see
how a longish one plays out.} \LaTeX\ handles the numbering
and placement of these headings for you, when you use
the appropriate heading commands around the titles
of the headings.  If you want a sub-subsection or
smaller part to be unnumbered in your output, simply append an
asterisk to the command name.  Examples of both
numbered and unnumbered headings will appear throughout the
balance of this sample document.

\subsubsection{The N-Queens Puzzle}
Because the entire article is contained in
the \textbf{document} environment, you can indicate the
start of a new paragraph with a blank line in your
input file; that is why this sentence forms a separate paragraph.

\subsection{Related Work}
We have already seen several typeface changes in this sample.  You
can indicate italicized words or phrases in your text with
the command \texttt{{\char'134}textit}; emboldening with the
command \texttt{{\char'134}textbf}
and typewriter-style (for instance, for computer code) with
\texttt{{\char'134}texttt}.  But remember, you do not
have to indicate typestyle changes when such changes are
part of the \textit{structural} elements of your
article; for instance, the heading of this subsection will
be in a sans serif\footnote{A third footnote, here.
Let's make this a rather short one to
see how it looks.} typeface, but that is handled by the
document class file. Take care with the use
of\footnote{A fourth, and last, footnote.}
the curly braces in typeface changes; they mark
the beginning and end of
the text that is to be in the different typeface.

You can use whatever symbols, accented characters, or
non-English characters you need anywhere in your document;
you can find a complete list of what is
available in the \textit{\LaTeX\
User's Guide}\cite{Lamport:LaTeX}.




%%%%%%%%%%%%%%%%%%%%%%%%%%%%%%%%%%%%%%%%%%%%%%%%%%%%%%%%%%%%%%%%%
% 
% APPROACH
%
%%%%%%%%%%%%%%%%%%%%%%%%%%%%%%%%%%%%%%%%%%%%%%%%%%%%%%%%%%%%%%%%%
\section{Our Approach}

- Describe the variable mutation rate algorithm and the two parameters that it
  has in comparison to fixed mutation's one param, (step size for changing the
  mutation rate and the inbreeding threshold).


\subsection{Applying Variable Mutation Rates}

- For explaining the variable mutation rate there is no need for a formal algorithm
  essentially it boils down to the following:

  - For each generation evaluate the population chromose similarity (see below
    for population chromosome similiarity algorithm)

  - If the population chromosome similarity is less than the threshold increase
    the mutation rate by the step size. If the similarity is greater than the
    threshold decrease the mutation rate by the step size. This results in
    the mutation rate adjusting up or down such that the population chromosome
    similarity approaches an equilibrium near the inbreeding threshold.


A formula that appears in the running text is called an
inline or in-text formula.  It is produced by the
\textbf{math} environment, which can be
invoked with the usual \texttt{{\char'134}begin. . .{\char'134}end}
construction or with the short form \texttt{\$. . .\$}. You
can use any of the symbols and structures,
from $\alpha$ to $\omega$, available in
\LaTeX\cite{Lamport:LaTeX}; this section will simply show a
few examples of in-text equations in context. Notice how
this equation: \begin{math}\lim_{n\rightarrow \infty}x=0\end{math},
set here in in-line math style, looks slightly different when
set in display style.  (See next section).


% This could be combined with the previous subsection
%\subsubsection{How Adaptation Occurs}
%A numbered display equation -- one set off by vertical space
%from the text and centered horizontally -- is produced
%by the \textbf{equation} environment. An unnumbered display
%equation is produced by the \textbf{displaymath} environment.


% This could be called algorithm, essentially explaining chromosome similarity
% algroithm since the concept of applying variable mutation can be easily explained in words
%\subsubsection{Solution Architecture}
\subsubsection{Chromosome Similarity Algorithm}

- For explaining the chromosome similarity algorithm (which is a little more
  complicated to avoid the naiive O(n!) solution, instead it is now linear
  complexity). Even though in terms of theoretical big-O complexity it is
  constant (since it's just based on the population size which is constant) it
  is worthwhile mentioning that from our empirical evidence using profiling of
  the code using the naiive O(n!) solution has a significant impact for larger
  population sizes (so much so that it caused a significant degrade in performance
  motivating us to profile the code and improve the original algorithm)


\begin{algorithm}
  \SetKwProg{Fn}{Function}{}{end}\SetKwFunction{Similarity}{Similarity}%
  \SetKwData{Similar}{similar}\SetKwData{Value}{value}\SetKwData{Matched}{matched}\SetKwData{Length}{length}\SetKwArray{Sorted}{sorted}
  \SetKwFunction{Sort}{Sort}
  \SetKwInOut{Input}{input}\SetKwInOut{Output}{output}
  \SetAlgoLined
  \DontPrintSemicolon
  
  \Fn{\Similarity{$chromosomes$}}{
  
  \Input{An array of $chromosomes$}
  \Output{Fraction of chromosomes that are similar}
  \BlankLine
  
  \Similar $\leftarrow$ 0\;
  \Matched $\leftarrow$ false\;
  \Length $\leftarrow$ length of $chromosomes$\;
  \BlankLine
  
  \tcp{Sort using an arbitrary sorting algorithm}
  \Sorted $\leftarrow$ \Sort{$chromosomes$}\;
  \BlankLine
  
  \For{$i \leftarrow 0$ \KwTo $\Length -1$}
  {
    \uIf{\Sorted{$i$} == \Sorted{$i + 1$}}
    {
      \Similar $\leftarrow$ \Similar + 1\;
      \Matched $\leftarrow$ true\;
    }
    \ElseIf{\Matched}
    {
      \Similar $\leftarrow$ \Similar + 1\;
      \Matched $\leftarrow$ false\;
    }
    \BlankLine
    
    \tcp{Case where the last item is a match}
    \If{\Matched and $\left(i + 1 == \Length - 1 \right)$}
    {
      \Similar $\leftarrow$ \Similar + 1\;
    }
  }
  \BlankLine
  \KwRet{\Similar / \Length}
  }
\caption{Chromosome similarity function}
\end{algorithm}



Again, in either environment, you can use any of the symbols
and structures available in \LaTeX; this section will just
give a couple of examples of display equations in context.
First, consider the equation, shown as an inline equation above:
\begin{equation}\lim_{n\rightarrow \infty}x=0\end{equation}
Notice how it is formatted somewhat differently in
the \textbf{displaymath}
environment.  



% I thought it would be important to add a section on implementation
% this could be condensed, but I think it is very important to
% explain how we actually implemented and applied the GA process
\subsection{Implementation}

- An overview of our GA implementation so that someone wishing to replicate our 
  results knows how we implemented our solution. 

- Could potentially provide a URL to the source code implementation on github

\subsubsection{Chromosome Design}
- Describe the chromosome design, the ordering of each gene of the chromosome represents
  the horizontal position of each queen on the board and the corresponding vertical
  position of the queen is stored in the gene of the chromosome.

  --> I think it would be wortwhile to have a very small diagram highlighting this

\subsubsection{Fitness Function}
- Fitness function, I can write the formal algorithm for this, each pair of queens 
  that have a vertical or diagonal collision is recorded as a value of 2 (one 
  collision from the perspective of each queen). This number is then represented 
  as a fraction over 1 such that a chromosome with a larger number of collisions 
  will have a much lower fitness value.
  

\begin{algorithm}
  \SetKwProg{Fn}{Function}{}{end}\SetKwFunction{Fitness}{Fitness}%
  \SetKwData{Collisions}{collisions}\SetKwData{Length}{length}\SetKwArray{Chromosome}{$chromosome$}\SetKwData{Yi}{$y_i$}\SetKwData{Yj}{$y_j$}
  \SetKwFunction{Abs}{abs}
  \SetKwInOut{Input}{input}\SetKwInOut{Output}{output}
  \SetAlgoLined
  \DontPrintSemicolon
  
  \Fn{\Fitness{$chromosome$}}
  {
    \Input{A single $chromosome$}
    \Output{A fitness value for the chromosome}
    \BlankLine
    
    \Collisions $\leftarrow$ 0\;
    \Length $\leftarrow$ length of the $chromosome$\;
    \BlankLine
    
    \For{$i \leftarrow 0$ \KwTo $\Length -1$}
    {
      \tcp{Check each gene against the current}
      $j \leftarrow \left( i + 1 \right) \mod{\Length}$\;
      \While{$j$ != $i$}
      {
        \Yi $\leftarrow \Chromosome{i}$\;
        \Yj $\leftarrow \Chromosome{j}$\;
        \BlankLine
        
        \tcp{Check for vertical collision}
        \If{\Yi == \Yj}
        {
          \Collisions $\leftarrow \Collisions + 1$\; 
        }
        \BlankLine
        
        \tcp{Check for diagonal collision}
        \If{\Abs{$\left(i - j \right)$ / $\left(\Yi - \Yj \right)$} == 1}
        {
          \Collisions $\leftarrow \Collisions + 1$\;
        }
        \BlankLine
        
        $j \leftarrow j + 1$\;
        $j \leftarrow j \mod{\Length}$\;
      }
    }
    \BlankLine

    \eIf{\Collisions == 0}
    {
      \KwRet{1}\;
    }
    {
      \KwRet{1 / \Collisions}\;
    }
  }
\caption{Fitness function}
\end{algorithm}




  
- If no collisions occur the default value is 1 resulting in a fitness of (1/1 = 1) 
  which is a  solution to the problem. 
  
- For queens with multiple collisions each pair of queens will be recorded as 2 collisions, 
  in the event that there are three pairs of queens with collisions, one pair with a vertical 
  collision, one with a diagonal, and one with a separate diagonal collision the fitness value 
  would be (1 / 6).
  
- In the event where there are multiple types of collisions (say two pairs of queens each with 
  vertical collisions, and one with a diagonal collision) the fitness value would be (1 / 8), 
  for a higher number of collisions this often happens making the fitness value even lower than 
  just a multiple of 2 x pairs of collisions.

  --> Might be helpful to give a diagram

  --> Due to the ordering of the queens in the chromosome there can only be
      vertical or diagonal collisions (since two queens can never be on the
      same row).

\subsubsection{Selection Method}
- Roulette wheel selection method, by generating a random floating point number and selecting
  a chromosome value where the random number lies within the bounds. The range
  of random values that can be chosen changes based on the maximum upper and lower
  bounds of the sum of the fitness of the chromosomes in the population.

- Each of the ranges of values that a particular chromosome can have is weighted
  based on the fitness of the chromosome. For example in a population with 4
  chromosomes two of which are solutions (fitness 1) and two with a single pair of collisions
  (fitness 1/2 = 0.5) the ranges of values for each chromosome would be as
  follows:

  [0, 1), [1, 2), [2, 2.5), [2.5, 3), and the bounds of the random floating point value
  generated for the roulette wheel selection would be [0, 3).


\subsubsection{Chromosome Operations}
- Crossover operation (70\% chance), cloning (30\% chance), (cite paper justifying why
  we used these values, they are often recommended/used values)

- Background mutation operation, applied to chromosomes, this is variable rather
  than a traditional fixed value, e.g. a 5\% value results in a 5\% chance of
  mutation being applied to chromosomes.

- Mutation operator changes ONE of the genes of the chromosome randomly, which
  results in changing the the y coordinate of a random queen in the chromosome 
  to a random value within the range of possible y values.


\subsubsection{Chromosome Evaluation}
- If a chromsome of the current population has a fitness value of 1 it is compared 
  to the list of previous solutions to see if it is unique. If the solution is 
  unique the rotation (rotating the solution an additional three times) and 
  reflection (performing reflection on the solution) followed by three additional
  rotations operations are applied to find a total of 8 solutions. Each solution 
  found after rotation and reflection is compared to the list of previous solutions to 
  verify that it is a unique solution.

  --> We do not keep duplicate solutions, this is important since someone may think
      that out of the 1000's of solutions we found many of them are just duplicates,
      when in fact they are all DISTINCT solutions.

- The current population is then replaced with the new population that was created
  by applying the crossover, cloning, and mutation operations.


% I would instead call this methodology
%\subsubsection{Testing and Validation}
\subsection{Methodology}

- Implementation in Java

- Experiments were conducted on the HPC facilities provided (SHARCNET)

- describe the study and control (diff. fixed mutation rates, list them all, vs.
  variable mutation rate with a fixed inbreeding threshold of 15\%), this had
  1 fixed param (inbreeding threshold)
  
  --> This should probably be put in a table.


- N queens problem sizes used: For 8 - 16 queens used each n queens size, for
  16 - 26 used each even sized N queens problem, lastly a test using 32 queens.

- Number of generations for each N queens problem (10 million generations for
  all except 32 queens), 50 million generations were used for 32 queens given
  the increased complexity of the problem.

- sample sizes (30, 15, 10)

    --> I will give you a table with each n queens problem, the sample sizes used 
        for each and the number of generations, this would reduce a lot of text
        needed to explain the items.

    --> justify why the sample sizes were reducedfor larger problems given the 
        computational limitations of SHARCNET (max 256 jobs, 7 days CPU time, 
        regardless and your job would be killed)

- describe the data collected, explain how each of the attributes such as
  population fitness, similarity are calculated based on the mean of the population
  for 1000 generations at a time (mean of means), rather than the mean of each
  population for each generation (TOO MUCH DATA!)



Now, we'll enter an unnumbered equation:
\begin{displaymath}\sum_{i=0}^{\infty} x + 1\end{displaymath}
and follow it with another numbered equation:
\begin{equation}\sum_{i=0}^{\infty}x_i=\int_{0}^{\pi+2} f\end{equation}
just to demonstrate \LaTeX's able handling of numbering.


% This really should be part of results, but could be left here
% to me it really doesn't fit in with approach since explaining
% our approach can't really justify yet "Why not fixed mutation"
\subsection{Why Not Fixed Mutation?}
Citations to articles \cite{bowman:reasoning,
clark:pct, braams:babel, herlihy:methodology},
conference proceedings \cite{clark:pct} or
books \cite{salas:calculus, Lamport:LaTeX} listed
in the Bibliography section of your
article will occur throughout the text of your article.
You should use BibTeX to automatically produce this bibliography;
you simply need to insert one of several citation commands with
a key of the item cited in the proper location in
the \texttt{.tex} file \cite{Lamport:LaTeX}.
The key is a short reference you invent to uniquely
identify each work; in this sample document, the key is
the first author's surname and a
word from the title.  This identifying key is included
with each item in the \texttt{.bib} file for your article.




%%%%%%%%%%%%%%%%%%%%%%%%%%%%%%%%%%%%%%%%%%%%%%%%%%%%%%%%%%%%%%%%%
% 
% RESULTS
%
%%%%%%%%%%%%%%%%%%%%%%%%%%%%%%%%%%%%%%%%%%%%%%%%%%%%%%%%%%%%%%%%%
% I would call this results, but performance analysis is good too
\section{Results}

- cite specific results and why we think they are important, what is the significance
  of them and how they support our results/hypothesis that in the case of N Queens
  var. mutation is better than fixed.

- cite the result showing the best fixed mutation vs. the variable mutation
  for each N-queens problem, try and use both the figure and the table, the table
  has some additional interesting information which cannot be conveyed in the
  image alone.

- cite interesting results of the fat boxplots in the range of chromosome similarity
  for the optimal fixed mutation rates, use the "person stepping" analogy and how
  that certain fixed mutation rates had the widest range in chromosome similiarty
  allowing it to hone in on solutions faster.

- Try and incorporate one of the scatter plots as well that show very interesting
  results and see if you can use it to compare/contrast the results of the 
  following plots
  
    - variable mutation rate scatter plot
    - variable mutation rate similarity scatter plot
    - best fixed mutation rate similarity scatter plot



The details of the construction of the \texttt{.bib} file
are beyond the scope of this sample document, but more
information can be found in the \textit{Author's Guide},
and exhaustive details in the \textit{\LaTeX\ User's
Guide}\cite{Lamport:LaTeX}.

This article shows only the plainest form
of the citation command, using \texttt{{\char'134}cite}.
This is what is stipulated in the SIGS style specifications.
No other citation format is endorsed or supported.




%%%%%%%%%%%%%%%%%%%%%%%%%%%%%%%%%%%%%%%%%%%%%%%%%%%%%%%%%%%%%%%%%
% 
% CONCLUSIONS
%
%%%%%%%%%%%%%%%%%%%%%%%%%%%%%%%%%%%%%%%%%%%%%%%%%%%%%%%%%%%%%%%%%
\section{Conclusions}

- Wait until the rest of the paper \& we have more feedback before
  writing the conclusions.

- Further research into adjusting the inbreeding threshold, for the purpose
  of the research a constant inbreeding threshold of 15\% was used, however
  further research could be done in testing different thresholds.

- Comparing the results of variable mutation with fixed mutation using other
  types of combinatorial/optimization problems such as TSP, constraint
  satisfaction problem (CSP), etc.

- Variable population size based on the amount of inbreeding (if you have
  a lot of inbreeding in nature the organisms will have higher mutatation
  rate and deformities, the population will shrink)

- Having the mutation operator affect more than one gene if it goes > 100\%? 

\begin{flushleft}\end{flushleft}

Because tables cannot be split across pages, the best
placement for them is typically the top of the page
nearest their initial cite.  To
ensure this proper ``floating'' placement of tables, use the
environment \textbf{table} to enclose the table's contents and
the table caption.  The contents of the table itself must go
in the \textbf{tabular} environment, to
be aligned properly in rows and columns, with the desired
horizontal and vertical rules.  Again, detailed instructions
on \textbf{tabular} material
is found in the \textit{\LaTeX\ User's Guide}.

Immediately following this sentence is the point at which
Table 1 is included in the input file; compare the
placement of the table here with the table in the printed
dvi output of this document.

\begin{table}
\centering
\caption{Frequency of Special Characters}
\begin{tabular}{|c|c|l|} \hline
Non-English or Math&Frequency&Comments\\ \hline
\O & 1 in 1,000& For Swedish names\\ \hline
$\pi$ & 1 in 5& Common in math\\ \hline
\$ & 4 in 5 & Used in business\\ \hline
$\Psi^2_1$ & 1 in 40,000& Unexplained usage\\
\hline\end{tabular}
\end{table}

To set a wider table, which takes up the whole width of
the page's live area, use the environment
\textbf{table*} to enclose the table's contents and
the table caption.  As with a single-column table, this wide
table will ``float" to a location deemed more desirable.
Immediately following this sentence is the point at which
Table 2 is included in the input file; again, it is
instructive to compare the placement of the
table here with the table in the printed dvi
output of this document.


\begin{table*}
\centering
\caption{Some Typical Commands}
\begin{tabular}{|c|c|l|} \hline
Command&A Number&Comments\\ \hline
\texttt{{\char'134}alignauthor} & 100& Author alignment\\ \hline
\texttt{{\char'134}numberofauthors}& 200& Author enumeration\\ \hline
\texttt{{\char'134}table}& 300 & For tables\\ \hline
\texttt{{\char'134}table*}& 400& For wider tables\\ \hline\end{tabular}
\end{table*}
% end the environment with {table*}, NOTE not {table}!

\subsection{Figures}
Like tables, figures cannot be split across pages; the
best placement for them
is typically the top or the bottom of the page nearest
their initial cite.  To ensure this proper ``floating'' placement
of figures, use the environment
\textbf{figure} to enclose the figure and its caption.

This sample document contains examples of \textbf{.eps}
and \textbf{.ps} files to be displayable with \LaTeX.  More
details on each of these is found in the \textit{Author's Guide}.

\begin{figure}
\centering
\epsfig{file=fly.eps}
\caption{A sample black and white graphic (.eps format).}
\end{figure}

\begin{figure}
\centering
\epsfig{file=fly.eps, height=1in, width=1in}
\caption{A sample black and white graphic (.eps format)
that has been resized with the \texttt{epsfig} command.}
\end{figure}


As was the case with tables, you may want a figure
that spans two columns.  To do this, and still to
ensure proper ``floating'' placement of tables, use the environment
\textbf{figure*} to enclose the figure and its caption.
and don't forget to end the environment with
{figure*}, not {figure}!

\begin{figure*}
\centering
\epsfig{file=flies.eps}
\caption{A sample black and white graphic (.eps format)
that needs to span two columns of text.}
\end{figure*}

Note that either {\textbf{.ps}} or {\textbf{.eps}} formats are
used; use
the \texttt{{\char'134}epsfig} or \texttt{{\char'134}psfig}
commands as appropriate for the different file types.


\subsection{Theorem-like Constructs}
Other common constructs that may occur in your article are
the forms for logical constructs like theorems, axioms,
corollaries and proofs.  There are
two forms, one produced by the
command \texttt{{\char'134}newtheorem} and the
other by the command \texttt{{\char'134}newdef}; perhaps
the clearest and easiest way to distinguish them is
to compare the two in the output of this sample document:

This uses the \textbf{theorem} environment, created by
the\linebreak\texttt{{\char'134}newtheorem} command:
\newtheorem{theorem}{Theorem}
\begin{theorem}
Let $f$ be continuous on $[a,b]$.  If $G$ is
an antiderivative for $f$ on $[a,b]$, then
\begin{displaymath}\int^b_af(t)dt = G(b) - G(a).\end{displaymath}
\end{theorem}

The other uses the \textbf{definition} environment, created
by the \texttt{{\char'134}newdef} command:
\newdef{definition}{Definition}
\begin{definition}
If $z$ is irrational, then by $e^z$ we mean the
unique number which has
logarithm $z$: \begin{displaymath}{\log e^z = z}\end{displaymath}
\end{definition}

Two lists of constructs that use one of these
forms is given in the
\textit{Author's  Guidelines}.
 
There is one other similar construct environment, which is
already set up
for you; i.e. you must \textit{not} use
a \texttt{{\char'134}newdef} command to
create it: the \textbf{proof} environment.  Here
is a example of its use:
\begin{proof}
Suppose on the contrary there exists a real number $L$ such that
\begin{displaymath}
\lim_{x\rightarrow\infty} \frac{f(x)}{g(x)} = L.
\end{displaymath}
Then
\begin{displaymath}
l=\lim_{x\rightarrow c} f(x)
= \lim_{x\rightarrow c}
\left[ g{x} \cdot \frac{f(x)}{g(x)} \right ]
= \lim_{x\rightarrow c} g(x) \cdot \lim_{x\rightarrow c}
\frac{f(x)}{g(x)} = 0\cdot L = 0,
\end{displaymath}
which contradicts our assumption that $l\neq 0$.
\end{proof}

Complete rules about using these environments and using the
two different creation commands are in the
\textit{Author's Guide}; please consult it for more
detailed instructions.  If you need to use another construct,
not listed therein, which you want to have the same
formatting as the Theorem
or the Definition\cite{salas:calculus} shown above,
use the \texttt{{\char'134}newtheorem} or the
\texttt{{\char'134}newdef} command,
respectively, to create it.

\subsection*{A {\secit Caveat} for the \TeX\ Expert}
Because you have just been given permission to
use the \texttt{{\char'134}newdef} command to create a
new form, you might think you can
use \TeX's \texttt{{\char'134}def} to create a
new command: \textit{Please refrain from doing this!}
Remember that your \LaTeX\ source code is primarily intended
to create camera-ready copy, but may be converted
to other forms -- e.g. HTML. If you inadvertently omit
some or all of the \texttt{{\char'134}def}s recompilation will
be, to say the least, problematic.

\section{Conclusions}
This paragraph will end the body of this sample document.
Remember that you might still have Acknowledgments or
Appendices; brief samples of these
follow.  There is still the Bibliography to deal with; and
we will make a disclaimer about that here: with the exception
of the reference to the \LaTeX\ book, the citations in
this paper are to articles which have nothing to
do with the present subject and are used as
examples only.
%\end{document}  % This is where a 'short' article might terminate

%ACKNOWLEDGMENTS are optional
\section{Acknowledgments}
This section is optional; it is a location for you
to acknowledge grants, funding, editing assistance and
what have you.  In the present case, for example, the
authors would like to thank Gerald Murray of ACM for
his help in codifying this \textit{Author's Guide}
and the \textbf{.cls} and \textbf{.tex} files that it describes.

%
% The following two commands are all you need in the
% initial runs of your .tex file to
% produce the bibliography for the citations in your paper.
\bibliographystyle{abbrv}
\bibliography{sigproc}  % sigproc.bib is the name of the Bibliography in this case
% You must have a proper ".bib" file
%  and remember to run:
% latex bibtex latex latex
% to resolve all references
%
% ACM needs 'a single self-contained file'!
%
%APPENDICES are optional
%\balancecolumns
\appendix
%Appendix A
\section{Headings in Appendices}
The rules about hierarchical headings discussed above for
the body of the article are different in the appendices.
In the \textbf{appendix} environment, the command
\textbf{section} is used to
indicate the start of each Appendix, with alphabetic order
designation (i.e. the first is A, the second B, etc.) and
a title (if you include one).  So, if you need
hierarchical structure
\textit{within} an Appendix, start with \textbf{subsection} as the
highest level. Here is an outline of the body of this
document in Appendix-appropriate form:
\subsection{Introduction}
\subsection{The Body of the Paper}
\subsubsection{Type Changes and  Special Characters}
\subsubsection{Math Equations}
\paragraph{Inline (In-text) Equations}
\paragraph{Display Equations}
\subsubsection{Citations}
\subsubsection{Tables}
\subsubsection{Figures}
\subsubsection{Theorem-like Constructs}
\subsubsection*{A Caveat for the \TeX\ Expert}
\subsection{Conclusions}
\subsection{Acknowledgments}
\subsection{Additional Authors}
This section is inserted by \LaTeX; you do not insert it.
You just add the names and information in the
\texttt{{\char'134}additionalauthors} command at the start
of the document.
\subsection{References}
Generated by bibtex from your ~.bib file.  Run latex,
then bibtex, then latex twice (to resolve references)
to create the ~.bbl file.  Insert that ~.bbl file into
the .tex source file and comment out
the command \texttt{{\char'134}thebibliography}.
% This next section command marks the start of
% Appendix B, and does not continue the present hierarchy
\section{More Help for the Hardy}
The sig-alternate.cls file itself is chock-full of succinct
and helpful comments.  If you consider yourself a moderately
experienced to expert user of \LaTeX, you may find reading
it useful but please remember not to change it.
%\balancecolumns % GM June 2007
% That's all folks!
\end{document}
