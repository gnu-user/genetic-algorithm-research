
\documentclass[conference]{IEEEtran}
\usepackage{stmaryrd}
\usepackage{amsfonts}
% If the IEEEtran.cls has not been installed into the LaTeX system files,
% manually specify the path to it: e.g.,
% \documentclass[conference]{../sty/IEEEtran}

\usepackage{graphicx,times,amsmath} % Add all your packages here
\usepackage{epstopdf}
\usepackage{algorithm2e}
\usepackage{multirow}
\usepackage[hidelinks]{hyperref}
% correct bad hyphenation here
\hyphenation{op-tical net-works semi-conduc-tor IEEEtran}

\IEEEoverridecommandlockouts    % to create the author's affliation portion
                % using \thanks

\textwidth 178mm    % <------ These are the adjustments we made 10/18/2005
\textheight 239mm   % You may or may not need to adjust these numbers again
\oddsidemargin -7mm
\evensidemargin -7mm
\topmargin -6mm
\columnsep 5mm

\begin{document}


% paper title: Must keep \ \\ \LARGE\bf in it to leave enough margin.
\title{\ \\ \LARGE\bf Enhancing the Performance of Genetic Algorithms in Combinatorial Optimization of Large and Difficult Problems\thanks{Daniel Smullen, Jonathan Gillett, Joseph Heron, and Shahryar Rahnamayan are with The University of Ontario Institute of Technology, Oshawa, Ontario, Canada (email: daniel.smullen@uoit.net).} \thanks{We would like to acknowledge the invaluable contribution made by our supervisor Dr. Shahryar Rahnamayan in transforming our junior year project into an interesting research paper. His advice and guidance has proved to be crucial to the success of our experiments. We would also like to thank the SHARCNET organization for graciously providing the high performance computing facilities we needed to run our experiments.}}

\author{Daniel Smullen, Jonathan Gillett, Joseph Heron, Shahryar Rahnamayan}

% avoiding spaces at the end of the author lines is not a problem with
% conference papers because we don't use \thanks or \IEEEmembership
% use only for invited papers
%\specialpapernotice{(Invited Paper)}

% make the title area
\maketitle

\begin{abstract}
Proposed is a novel methodology for optimization of genetic algorithms using adaptive variable mutation. We replace the traditional mutation operator with an adaptive operator governed by population chromosome similarity. This approach is inspired from biological genetic expression which has increased incidences of genetic mutations as a result of `inbreeding' - we abstractly model this process as a fixed-size 1\% increase in mutation likelihood over generations, based on a static 15\% chromosome similarity threshold. Using combinatorial optimization with GA to solve the N-Queen problem, we determine optimal static mutation parameters for the traditional GA approach to use as a benchmark. We compare our approach to the benchmark for N = 8 through 32. Comparison between these optimized fixed mutation parameters and our adaptive variable approach shows better results using our approach for large problems (N > 15), with improving results for increasing values of N. A near tenfold increase in performance is yielded for N = 32.
\end{abstract}

% no key words

\section{Introduction}


\PARstart{G}{enetic algorithms} (GA) serve an important role in various applications, particularly useful in areas which they can perform stochastic generation of solutions for combinatorial problems \cite{de1989using,crawford1992solving}. One well-known example of this is the N-Queen problems. One approach to solve the N-Queen problems using a genetic algorithm is to implement chromosomes which represent the position of each queen on the chessboard. The permuted values stored in the chromosome represent the sequential row positions of each queen, encoded into binary values. Each column value is the index, since there can never be more than one queen per column. Thus, the column positions increase by one for each queen and the intersections (referred to as collisions) are calculated on the diagonal to determine whether a solution has been found or not. As with any genetic algorithm, evaluating the fitness of the chromosomes is required to determine the quality of candidate solutions \cite{srinivas1994genetic}. 

Tuning up control parameters has a crucial effect on the GA's performance, which can increase/decrease based on the landscape of the problem encountered \cite{ye2010some,coyne1994genetic,srinivas1994adaptive}. We have found that by replacing a static mutation value with a self-adaptive one which is controlled by chromosome similarity, better performance can be achieved. This is compounded when the search space is extremely huge, such as the situation encountered when higher-order N-Queen problems.


\section{Background}
Like many other complex optimization problems, the N-Queen problem becomes orders of magnitude more complex as the number of queens and the size of the chessboard increases \cite{homaifar1992queens}. When two queens share the same diagonal, a collision occurs, meaning that the board state is not a solution.

The 8-Queen problem is the classical version of N-Queen. Even with this simplistic configuration, the problem is computationally expensive. There are {$64 \choose 8$} arrangements of the queens on a standard {$8\times{}8$} board. An exhaustive deterministic evaluation has shown that there are only 92 distinct solutions. Even including protective embedded constraints on placing only one queen per row/per column, there are still {$8!$} possible distinct arrangements of queens. In higher-order versions of the problem, a combinatorial explosion occurs. The 9-Queen problem has 352 distinct solutions. The 10-Queen problem has 724. Table \ref{table:numuniquesol} summarizes the growth of the number of solutions toward intractability. 

While the number of solutions is known for problems involving up to 26 queens, there is currently no known formula to determine the exact number of distinct solutions for {$N>26$} problems. That is to say, higher order N-Queen problems cannot be solved for all solutions exhaustively. Deterministic methods toward tackling these problems are useless \cite{crawford1992solving}.

\subsection{The N-Queen Puzzle}
The natural question raised when approaching simple problems is why an exhaustive deterministic strategy is not used. Unfortunately, these approaches cannot work within human time-frames except in small problem sizes. The issue is that a direct linear traversal of the solution landscape does not yield fruitful results easily in the case of the N-Queen problem. This stems from the fact that the relative fitness of each sequential board state doesn't matter unless it is a valid solution. The problem cannot be solved by producing a solution that has queens which will attack each other. Therefore, only solutions with the absolute maximum, 100\% fitness, are useful solutions \cite{crawford1992solving}. As the problem landscape increases in size, the number of distinct solutions does not increase proportionally. A full brute force traversal of the landscape would be required to find all the distinct solutions, which is increasingly expensive in higher order problems.

Other stochastic solutions are impractical with this type of problem. This is because finding solutions among the landscape of possible board states yields a tremendous amount of duplicate solutions. Some method is required to direct the random traversal of the landscape toward distinct solutions and away from those which have already been found. The problem with this statement is that the nature of randomness cannot be constrained in this way - it will no longer be random. Therefore, methods such as genetic algorithms become an attractive solution approach  because they represent a 'smarter' traversal of the problem landscape than what is offered by a linear brute force traversal, or a uniformly random traversal of the landscape\cite{srinivas1994genetic}. Less time is spent on areas of the landscape which are not likely to be solutions. This leads us toward the core motivation of our solution's behavior.

\subsection{Motivation}\label{motivation_section}

Deterministic modeling of the randomness of the solution landscape as seen in the N-Queen problem is impossible with known mathematical methods. To produce a geometric solution landscape would require that the dependent axis of the coordinate plane maps each sequential value to a uniformly random value in the actual solution landscape. Conceptually this is very difficult to imagine, but the reason why this conceptual strategy is useful will become clearer - first, we will discuss why a new approach is necessary in comparison to traditional GA approaches.

There are several optimizations to generating solutions which are available when approaching the N-Queen problem because of the symmetrical nature of the chess board. Since it is always square, one solution can be used to yield many other distinct solutions using symmetry operations (reflection and rotation of the board). This means that up to 8 solutions can be generated by finding one distinct solution. However, symmetry operations cannot be used to find all distinct solutions because many symmetrical configurations of the board are equivalent. In the example of the Eight-Queens problem, there are 11 board configurations which yield 8 solutions, and 1 which yields only 4. The result is still 92 distinct solutions, with the majority of them calculable through symmetry operations alone.

Using symmetry operations, our initial exploration into solving the N-Queen problem yielded solutions quickly, but not quickly enough to beat deterministic solution generation methods. This was especially apparent in lower order problems where the solution landscape is small. We began to explore biological nuances of genetics in relation to genetic algorithms, and sought inspiration from the fact that nature has an apparent system for maintaining genetic variation within species. For example, pure-bred dogs often tend to have significant health problems which directly result from inbreeding \cite{clark1983medical}. Royal families historically used inbreeding to keep their line pure, which also resulted in genetic abnormalities\cite{mannucci2001hemophilias}.

These facts led us toward a potential approach to apply the effects of inbreeding towards the optimization of genetic algorithms. Our approach would make the mutation rate variable based on chromosome similarity, evoking the increased mutations seen inbreeding as a result of a too-similar biological chromosome. Our initial solution worked exceptionally well, providing all 92 solutions nearly instantaneously in comparison to our previous static mutation rate GA which required several seconds. This motivated us to explore the implications of our new strategy further.

\section{Related Work}
Classical genetic algorithms have been used extensively as a means to perform optimization in engineering for many years \cite{goldberg1994genetic,srinivas1994genetic,de1989using}. A popular problem to benchmark optimization algorithms in general is to use the N-Queen problem \cite{homaifar1992queens,crawford1992solving}. However, as stated in \cite{crawford1992solving}, the N-Queen problem is very different from most NP-Complete problems. This stems from the fact that in spite of the fitness functions used in the algorithm, solutions ultimately evaluate to a binary value. This prevents a directed search through the solution landscape since there is no feedback besides the board state found being absolutely correct or incorrect. 

One major concern with using classical GA is determining the best parameters for genetic operators. That is to say, the probability of mutation and crossover. These operators' values are defined prior to the GA execution and remain static for the entire execution. Hesser and M\"{a}nner's investigations define an algorithmic solution providing suitable, optimized values for genetic operators \cite{hesser1991towards}. Their approach negates the need for determining the best parameters, but adds to the complexity of the problem. This was an important consideration for our approach.

Other unconventional attempts to improve the selection of GA operations have been made over the last few decades. Since GA are inspired by real life biological behaviors, adjustments tend to be based on adding augmentations describing natural phenomena to improve the GA performance. One major example of this is Adaptive Genetic Algorithms (AGA). AGA attempts to adjust the genetic operations actively throughout the evaluation of the GA to escape local optima and reach a maximum (or minimum) faster. Typical examples of AGA use a defined measurement of some performance metric of the population, such as the population's mean fitness, to determine when to adjust genetic operators \cite{coyne1994genetic,srinivas1994adaptive}. Because of this adaptability, the use of AGA requires less \emph{a priori} knowledge of the search space. Adaptation means that in some examples, as fitness increases, mutation rates decrease, or \textit{vice-versa}. 

In one specific example, Coyne and Paton only changed the mutation rate between two values; a 'hypermutable' (high) rate, and normal (typical) mutation rate \cite{coyne1994genetic}. Use of the population fitness at large provides the impetus to switch between the two values. Srinivas and Patnaik provide a similar case for adaptive mutation rates. However, their case study's genetic operators are not restricted to two values. Rather, they are calculated to suit the situation. A different approach to AGA is described by Tuson and Ross, in which each individual chromosome in a population has its own defined genetic operators. The genetic operators will be adapted and changed during the creation of the next generation \cite{tuson1998adapting}. 

Ye, Li and Xie give a fundamental insight into adaptive genetic algorithms by comparing two major, contradictory processes: mutation-first, and crossover-first. Mutation-first adaptation relies on a high mutation rate and a low crossover rate initially. Over time, it will progress to the opposite - a low mutation rate and a high crossover rate. Crossover-first is the exact opposite from the beginning, with a low mutation and a high crossover rate initially. It was found that the mutation-first method outperformed the crossover-first \cite{ye2010some}. This can be attributed to the need to explore the search space early in the execution (through mutation) and to make smaller adjustments (through increased crossover operations) later on. An attempt was also made to provide a plausible methodology toward solving the local optimum convergence problem by introducing randomly generated individuals to the population. The intention was to increase diversity and the approach with some success. 

Our work seeks to provide an explanation for the behavior of our approach by utilizing these works in conjunction with an empirical study and a novel conceptual model for understanding the behavior of the algorithm.

\section{Proposed Approach}\label{params}
Ordinary genetic algorithms use three fundamental parameters to govern their evaluation. The first is the population size. We have not adjusted this parameter in our experiments, and kept them at a fixed value of 64 chromosomes as suggested by \cite{negnevitsky2005ai}. Second is the chromosome crossover rate, which governs the likelihood and percentage of a chromosome to splice with another. This was also kept static, at a fixed rate of 70\% \cite{negnevitsky2005ai}. The third parameter is mutation rate, which governs the likelihood that a chromosome will have a random gene modified. This is achieved using a randomization operation on the selected chromosome's encoded values.

To support the 'in-breeding' variant of the algorithm, we have selected to replace the fixed mutation rate parameter with a variable rate, governed by a separate parameter which was kept at a fixed value for the sake of minimizing variability in our experimental methodology. The new parameter, chromosome similarity threshold, is expressed as a percentage. We have selected 15\% as an arbitrary threshold which seemed to work well during initial experimentation. Future studies into the nature of this parameter may yet yield improved results. The offset by which the mutation rate is adjusted (based on the chromosome similarity parameter) is set at a fixed value of 1\%.

\subsection{Applying Adaptive Variable Mutation}
In order to apply our adaptive variable mutation approach, a two step process is required. For each generation of chromosomes, the population's chromosome similarity must be evaluated. The similarity algorithm is detailed in Algorithm \ref{alg:similarity}.  

When the chromosome similarity is less than the specified threshold, the population is dissimilar, and inbreeding does not come into play. This means the mutation rate must decrease, because the population is not inbred. For the next generation, the mutation rate is decreased by the fixed value of 1\%. The next generation is created and their chromosome similarity is calculated. If the chromosome similarity has increased again, and it is above the specified 15\% threshold, the population is inbred. The mutation rate must increase. 

Due to the stochastic nature of the algorithm, variations in chromosome similarity may result regardless of the mutation rate. Since the selected crossover rate of the genes is 70\%, the resultant permutations of the genome are likely to be approximately 30\% similar (due to cloning) unless mutation occurs on cloned chromosomes. The result of this adaptive change in mutation rate is that the chromosome similarity will approach an equilibrium close to the specified inbreeding threshold over generations.

\subsection{Chromosome Similarity Algorithm}
Chromosome similarity is calculated using a greedy algorithm seen in Algorithm \ref{alg:similarity}. First, the chromosomes' genes are decoded and re-encoded into integer values, concatenated into one large integer value. next, the array of chromosomes' concatenated integer representations is sorted. Our implementation uses the quick-sort algorithm provided by Java for minimal computational overhead. This gives the algorithm its' characteristic asymptotic behavior, as the complexity of the actual sorting function itself is of higher complexity than the main similarity calculation algorithm. Quick-sort runs on average in $O(n \log n)$ complexity, although it should be noted that in the theoretical worst case, quick-sort works in $O(n^2)$ which is highly unlikely. The main algorithm runs in $O(n)$ linear complexity independent of the sorting.

\begin{algorithm}
  \SetKwProg{Fn}{Function}{}{end}\SetKwFunction{Similarity}{Similarity}%
  \SetKwData{Similar}{similar}\SetKwData{Value}{value}\SetKwData{Matched}{matched}\SetKwData{Length}{length}\SetKwArray{Sorted}{sorted}
  \SetKwFunction{Sort}{Sort}
  \SetKwInOut{Input}{input}\SetKwInOut{Output}{output}
  \SetAlgoLined
  \DontPrintSemicolon
  
  \Fn{\Similarity{$chromosomes$}}{
  
  \Input{An array of $chromosomes$}
  \Output{Fraction of chromosomes that are similar}
  \BlankLine
  
  \Similar $\leftarrow$ 0\;
  \Matched $\leftarrow$ false\;
  \Length $\leftarrow$ length of $chromosomes$\;
  \BlankLine
  
  \tcp{Sort using an arbitrary sorting algorithm}
  \Sorted $\leftarrow$ \Sort{$chromosomes$}\;
  \BlankLine
  
  \For{$i \leftarrow 0$ \KwTo $\Length -1$}
  {
    \uIf{\Sorted{$i$} == \Sorted{$i + 1$}}
    {
      \Similar $\leftarrow$ \Similar + 1\;
      \Matched $\leftarrow$ true\;
    }
    \ElseIf{\Matched}
    {
      \Similar $\leftarrow$ \Similar + 1\;
      \Matched $\leftarrow$ false\;
    }
    \BlankLine
    
    \tcp{Case where the last item is a match}
    \If{\Matched and $\left(i + 1 == \Length - 1 \right)$}
    {
      \Similar $\leftarrow$ \Similar + 1\;
    }
  }
  \BlankLine
  \KwRet{\Similar / \Length}
  }
\caption{Chromosome similarity function}
\label{alg:similarity}
\end{algorithm}

\subsection{Implementation Specific Considerations}
In order to detail how our specific implementation might compare to alternatives, we will elaborate upon the special implementation considerations we made. Our production experiments were performed our development code, which was written in Java. Concurrent agile development was used as our main software development methodology, and GitHub was used extensively for version management and concurrent development of the source, documentation, and this report. 

The project source code is available at \url{http://www.github.com/gnu-user/genetic-algorithms-research}

\subsection{Fitness Function}
Fitness is evaluated by determining the number of collisions between queens on the chessboard - when at least two queens can attack each other, a collision occurs for each queen. This means that in a board state where one queen can attack another, two collisions result. Because it is impossible for there to be more than one queen per column given the chromosome's encoded data structure, only collisions across rows and diagonals must be calculated to determine fitness. The algorithm for determining the fitness of a given board state is given in Algorithm \ref{alg:fitness}, showing the mechanisms we used to find collisions across rows and diagonals. The overall board state fitness is calculated as $fitness = \frac{1}{C}$, where $fitness = 1$ if $C = 0$. The number of collisions is $C$. Maximum fitness is achieved when $C = 0$ as there are no collisions, resulting in a distinct board solution. The length of the chromosome corresponds to the number of queens and rows on the square chessboard, $N$. As a result, the fitness function must check across $N - K$ rows and diagonals for each queen, where $K$ is the column index of the current queen.
 
\begin{algorithm}[t!]
  \SetKwProg{Fn}{Function}{}{end}\SetKwFunction{Fitness}{Fitness}%
  \SetKwData{Collisions}{collisions}\SetKwData{Length}{length}\SetKwArray{Chromosome}{$chromosome$}\SetKwData{Yi}{$y_i$}\SetKwData{Yj}{$y_j$}
  \SetKwFunction{Abs}{abs}
  \SetKwInOut{Input}{input}\SetKwInOut{Output}{output}
  \SetAlgoLined
  \DontPrintSemicolon
  
  \Fn{\Fitness{$chromosome$}}
  {
    \Input{A single $chromosome$}
    \Output{A fitness value for the chromosome}
    \BlankLine
    
    \Collisions $\leftarrow$ 0\;
    \Length $\leftarrow$ length of the $chromosome$\;
    \BlankLine
    
    \For{$i \leftarrow 0$ \KwTo $\Length -1$}
    {
      \tcp{Check each gene against the current}
      $j \leftarrow \left( i + 1 \right) \mod{\Length}$\;
      \While{$j$ != $i$}
      {
        \Yi $\leftarrow \Chromosome{i}$\;
        \Yj $\leftarrow \Chromosome{j}$\;
        \BlankLine
        
        \tcp{Check for vertical collision}
        \If{\Yi == \Yj}
        {
          \Collisions $\leftarrow \Collisions + 1$\; 
        }
        \BlankLine
        
        \tcp{Check for diagonal collision}
        \If{\Abs{$\left(i - j \right)$ / $\left(\Yi - \Yj \right)$} == 1}
        {
          \Collisions $\leftarrow \Collisions + 1$\;
        }
        \BlankLine
        
        $j \leftarrow j + 1$\;
        $j \leftarrow j \mod{\Length}$\;
      }
    }
    \BlankLine

    \eIf{\Collisions == 0}
    {
      \KwRet{1}\;
    }
    {
      \KwRet{1 / \Collisions}\;
    }
  }
\caption{Fitness function}
\label{alg:fitness}
\end{algorithm}

\subsection{Selection Method}
As in traditional GA, roulette wheel selection was used \cite{goldberg1991comparative}. This process works by filling the roulette wheel with fractional parts of a cumulative probability distribution. In our implementation the range of values will lie between 0 and the sum of all the fitness values of each chromosome in a generation.

Each chromosome is given a fraction of the roulette wheel, whose size is equal to the chromosome's fitness divided by the total fitness of all values. A random floating point value is generated between zero and the sum of the total fitness values. This floating point value will lie within one of the intervals representing one of the chromosomes. Whichever interval it falls in is the selected chromosome. The order in which the chromosomes' intervals are placed doesn't matter, as each 'spin' of the roulette wheel is random - the floating point number generated is created based on a uniformly random distribution.

\subsection{Genetic Operations}
As mentioned in section \ref{params}, the crossover rate is set at a fixed 70\%. This means that 30\% of the time, the genes in the child chromosomes will be an exact clone of the parents. The remaining times, a random crossover occurs. This is done using the traditional GA approach by selecting a random index as the bounds for the portion of the chromosome to cross over. All genes to the right of this index (including the index value itself) are swapped between parents to generate offspring. Between the two parents, the first parent's values to the right of the index, and the second parent's values to the left of the index are combined to create one child (and \textit{vice-versa} for the second child).

Background mutation affects chromosomes randomly after all crossover operations take place. Each next-generation chromosome has a percentage chance of being affected by mutation, and this likelihood changes every generation due to the adaptive variable mutation rate. When a chromosome is selected for mutation, one of the genes in the chromosome is selected randomly. It is then set to a random valid integer. Valid values are between 0 and N corresponding to the index of each row on the chessboard.

\subsection{Chromosome Evaluation}
Each generation, there is a likelihood that some of the chromosomes in the current population are valid solutions - any of the current generation's chromosomes with a fitness of 1 is some sort of solution. Whether it is distinct or not is unknown at this stage. When a solution is found, it must be compared to the list of previously found solutions to ensure it is not a duplicate. Due to the stochastic nature of the algorithm, there is always a strong possibility that many solutions generated are duplicates of previously found solutions. The symmetry operations applied to each solution found may further increase the likelihood that the new population's solutions that have already been found. Duplicate solutions are discarded without having symmetry operations applied as they would already be identical to symmetric configurations of previously found solutions.

Upon saving any newly found solutions, the new generation replaces the previous generation and the genetic algorithm process cycle continues again.

\section{Experimental Methodology}
Experiments were conducted using the high performance computing facilities provided by SHARCNET. Testing of our process was performed by using two different configurations for each N value of the N-Queen problem. Tests were performed for $N = 8, 9, 10$ and so on up to 16, and then at increasing increments of 2 from 18 to 20, 22, 24, 26 and 32. A fixed number of generations was allowed for each trial in order to ensure that the tests had a finite and equal running time. For all runs, 10 million generations were allowed, except for $N = 32$, where 50 million generations were allowed given the increased complexity of the problem.

The first configuration used static mutation rates set at $m = 1\%, 5\%, 10\%$, and every further increment of 5\% up to 100\%. These were used as a control. The second configuration used the adaptive variable mutation rate, and the results were compared. 

For $N = 8$ through 16, 30 tests were performed at each static mutation rate ($21 \times 30 = 630$ tests total), and then 30 runs were performed using variable mutation rates. 

For $N = 18$ through 26, 15 tests were performed under both configurations. For $N = 32$, 10 tests were performed. The decrease in the number of tests performed for higher values of N was largely due to the running time required to generate meaningful results. SHARCNET provisioned us to execute a maximum of 256 jobs at one time, with 7 days CPU time each. The 32-Queens problem required this limit to be fully utilized in order to give results. Given the intractable size of the higher-order N-Queen problems beyond 32-Queens, an unsustainable amount of time and CPU power would be required to finish these problems within our time constraints and were not tested.

Given the volume of data generated by the number of trials, statistical attributes of the results collected (such as mean population fitness and chromosome similarity) were calculated using the mean of every 1000 generations in order to observe the central tendency. A total of roughly 6 gigabytes of data was generated as a result of this process. This was manageable given our allocation of storage space on SHARCNET. The requirements for collecting the data at every generation would have been 1000 times that volume (approximately 6 terabytes) which far exceeded our allocation.

\section{Results and Analysis}
It became clear very quickly when we plotted the results of our tests that the variable mutation method was not suitable in its current configuration for the simpler N-Queen problems. In fact, we found that given sufficient knowledge about the problem landscape for the lower-order problems, the right static mutation value would likely beat variable mutation every time. However, for ordinary non-benchmark usage of genetic algorithms it is useless to use a static mutation rate value unless you already know the best mutation rate for the problem. This generally speaks to the point that having an adaptive mutation rate approach will likely find this value. The smaller N-Queen problems are too small to merit the usage of a stochastic approach regardless, as deterministic methods can yield solutions quickly with guaranteed success.

\subsection{Traversing the Problem Landscape}
Moving into more complex and larger problems is where the adaptive mutation approach becomes increasingly successful. Our reasoning is that this may have something to do with the similarity threshold value that was chosen (15\%). Further tuning of this parameter may yield better results for larger or smaller problems. We have created a conceptual model for our findings which may yet be validated in future studies.

Mapping the stochastic solution landscape for the N-Queen problem means plotting a random non-deterministic traversal of the solution space to a linear axis. Each sequential value on this axis corresponds to each integer sequence of solutions encoded in the chromosomes. A linear traversal of this landscape implies a uniformly random traversal of the randomly generated solutions. This plot, in contrast to the deterministic solution space yielded by consecutively plotting the values of the integer encoded chromosome, shows very few similarities. There is no way to plot a random traversal except by noting that as time approaches infinity, all solutions will have been traversed and the full contour of the landscape will be visible. This could just as easily be yielded by a consecutive linear traversal, but the point of using GA rather than a deterministic traversal is that GA makes the traversal 'smarter' by searching for values which have better fitness, attracted toward fitness peaks\cite{srinivas1994genetic}. The mutation operator helps to prevent the current position in the traversal from being stuck in a local optimum \cite{ye2010some}. It provides an extra impulse to shift position further towards what a potential solution, and away from previously traversed areas of the landscape.

Given this knowledge, assuming a linear traversal of this random axis is possible in one given test, we can evaluate the characteristics of each GA mutation parameter configuration and map them to this conceptual model. The starting point on the plot is a random location on the landscape. From there, the next location is likely to be within a radius of probable new chromosomes surrounding this area. We have come to refer to this radius of probable chromosomes as the 'step size', and we believe there is some relationship between the step size and the fixed mutation rate. The fixed mutation rate governs the maximum distance from the current position in the traversal to the furthest step away. What this means for a solution landscape like that of the N-Queen problem is that all solutions which are close to the intersections of these steps are easier to find because there is a greater total probability of landing within their vicinity. Solutions which are clustered very close together on this random landscape, however, will likely be stepped around for several generations before convergence on the actual solution occurs.

The variable mutation rate's adaptive nature allows for this fixed step size to be ignored because it introduces the ability to vary the step size when convergence upon a solution occurs. The result is that the probability distribution over the entire solution landscape is more uniformly smaller away from solutions, and larger within the vicinity of solutions. Our reasoning is that if the static mutation rates have a fixed step size, the variable mutation rate does not, and adaptation will vary the step size so as to maintain the chromosome similarity. Further, this decreases the likelihood of overshooting a solution and having to step back around again, because the variability of the mutation rate allows for smaller or larger steps when needed.

In general, this conceptual model will require extensive further study to validate it's correctness. Meanwhile, the empirical evidence presented in our plotted results shows the validity of our adaptive mutation approach.

\subsection{Plotted Results and Observations}
Figure \ref{fig:best_solution_all_queens} depicts the best fixed mutation rate, that found the most solutions for each problem size, versus the variable mutation rate. Each of these plots are cut off at 10 million generations. In spite of the fact that the 32-Queens problem was allowed to run up to 50 million generations, we are only showing plots which are trimmed to this value for consistency of comparison. The plots show that as the problem size increases from the smaller problems up to 14 queens, the fixed mutation rates reach peak performance. In these problems, the most optimal fixed mutation rate is better than the variable mutation rate. This trend changes at the 15 queens where they are nearly equal. From there on, the difference in performance becomes significant - variable mutation consistently finds more solutions than the most optimal fixed mutation rate. 

The optimal fixed mutation rate decrease from 95\% for 8 queens down to 65\% for 32 queens. As the problem size increases, we observe that this trend continues. This can be seen in Table \ref{table:bestsol}. From 15 queens up to 32 queens, there is a divergence between the performance of fixed mutation and variable mutation. A steep decline is visible in the performance of the best fixed mutation rates as the problem size increases. The performance of the variable mutation rate also decreases with increased problem sizes, but at a far lesser rate. This marginal decrease in performance yields the conclusion that variable adaptive mutation rates will perform far better for larger problems.

An interesting result was observed when viewing the box plots showing the range of chromosome similarity values for each static mutation rate. We found that whenever the similarity has the greatest inter-quartile range (IQR), the corresponding static mutation rate performed best. This contributed heavily to the step size concept we believe is exhibited by the system. We also found that as the problem landscape grows larger in higher order N-Queen problems, the IQR decreases in size. There is a possibility that the IQR will decrease towards convergence on a single value in higher order problems, which suggests that there may be a pattern to the convergence which can be approximated and leveraged for predicting optimal static mutation rates for any N-Queen problem. In the 14-Queens problem, the IQR was approximately 20\%. In the 32-Queens problem, the IQR was approximately 3\%. Since testing this requires exhaustive experimentation in order to definitively find the optimal static mutation rate for any given problem, this further supports the notion that a variable mutation rate is a more practical approach. As shown prominently in Figure \ref{fig:fitness_all_mutation_14}, we have found that the relationship between the best fixed mutation rates and the largest IQR is visible. The fixed mutation rate of 80\% clearly has the largest IQR, and correspondingly the most number of solutions found. This trend is visible for all values of N in the N-Queen problem. 

However, when approaching problem sizes of $N > 14$, the observation no longer holds true - adaptive variable mutation rates are consistently better, but have comparatively similar IQR to the worse performing static mutation rates. We noticed a new relationship here, showing that the IQR of the variable mutation rate is closest to the trend followed by the IQR of the increasing static mutation rates. There is a sharp drop-off, however, immediately upon reaching the best static mutation rates within this trend. This supports the step-size concept in that we can see the inability of the static mutation rate to converge upon the most desirable fitness values. There is a clear convergence on this value with the adaptive mutation rate, consistently approaching the 25\% fitness value. The most optimal static mutation rate in our experimentation is higher than the true optimum, resulting in a sharp falloff of fitness values. In Figure \ref{fig:best_solution_all_queens}, the 14-Queens problem shows the best static mutation rate is 80\%, which performs substantially better than any other static mutation rate. In Figure \ref{fig:fitness_all_mutation_14}, this static mutation clearly has the largest IQR. In the 32-Queens problem, the 65\% fixed mutation rate is the best performer, but is beaten nearly tenfold in performance by the variable mutation rate.

Figure \ref{fig:mutation_rate_all_queens} shows a box plot with an overlaid trend line. The box plots in this figure show the IQR for the mutation rate of the variable adaptive method. The points show the best performing static mutation rate for each N-Queen problem. This plot shows the convergence trend of the IQR in the variable mutation rates as the problem size increases. There is also a trend that shows the best performing static mutation rates decrease as the problem size increases. Where the trend and the box plots intersect at the 15-Queens problem, we can see in the corresponding Figure \ref{fig:best_solution_all_queens} that the performance is nearly the same.


%%%%%%%%%%%%%%%%%%%%%%%%%%%%%%%%%%%%%%%%%%%%%%%%%%%%%%%%%%%%%%%%%
% 
% TABLES
%
%%%%%%%%%%%%%%%%%%%%%%%%%%%%%%%%%%%%%%%%%%%%%%%%%%%%%%%%%%%%%%%%%
\begin{table}[!h]
\centering
\caption{N-Queens Problem Distinct Solutions}
\begin{tabular}{|l|l|} \hline
N & Distinct Solutions      \\ \hline
8  & 92                     \\
9  & 352                    \\
10 & 724                    \\
11 & 2680                   \\
12 & 14,200                 \\
13 & 73,712                 \\
14 & 365,596                \\
15 & 2,279,184              \\
16 & 14,772,512             \\
17 & 95,815,104             \\
18 & 666,090,624            \\
19 & 4,968,057,848          \\
20 & 39,029,188,884         \\
21 & 314,666,222,712        \\
22 & 2,691,008,701,644      \\
23 & 24,233,937,684,440     \\
24 & 227,514,171,973,736    \\
25 & 2,207,893,435,808,352  \\
26 & 22,317,699,616,364,044 \\
\dots & Unknown				\\
\hline\end{tabular}
\label{table:numuniquesol}
\end{table}

\begin{table}[!h]
\centering
\caption{Fixed and Variable Mutation Comparison}
\begin{tabular}{|l|l|l|l|l|l|l|} \hline
N&               Mutation Rate&  Distinct Solutions\\ \hline
\multirow{2}{*}{8}&   0.95&           92\\
&                     variable&       92\\ \hline
\multirow{2}{*}{9}&   0.95&           352\\
&                     variable&       352\\ \hline
\multirow{2}{*}{10}&  0.9&            724\\
&                     variable&       724\\ \hline
\multirow{2}{*}{11}&  0.85&           2,680\\
&                     variable&       2,680\\ \hline
\multirow{2}{*}{12}&  0.85&           13,690\\
&                     variable&       11,986\\ \hline
\multirow{2}{*}{13}&  0.8&            32,128\\
&                     variable&       26,308\\ \hline
\multirow{2}{*}{14}&  0.8&            41,520\\
&                     variable&       29,520\\ \hline
\multirow{2}{*}{15}&  0.8&            30,356\\
&                     variable&       30,324\\ \hline
\multirow{2}{*}{16}&  0.8&            15,016\\
&                     variable&       30,132\\ \hline
\multirow{2}{*}{18}&  0.75&           16,392\\
&                     variable&       27,120\\ \hline
\multirow{2}{*}{20}&  0.75&           7,872\\
&                     variable&       25,608\\ \hline
\multirow{2}{*}{22}&  0.7&            6,008\\
&                     variable&       25,376\\ \hline
\multirow{2}{*}{24}&  0.7&            6,440\\
&                     variable&       24,560\\ \hline
\multirow{2}{*}{26}&  0.7&            6,280\\
&                     variable&       23,008\\ \hline
\multirow{2}{*}{32}&  0.65&           15,216\\
&                     variable&       104,080\\
\hline\end{tabular}
\label{table:bestsol}
\end{table}





\subsection{Why Not Fixed Mutation?}
Variable adaptive mutation gives a distinct advantage over traditional genetic algorithms because of the optimization of a specific parameter \cite{hesser1991towards,ye2010some}. By using a variable mutation rate, replacing the mutation parameter with a chromosome similarity, better results can be gleaned while knowing less about the problem landscape or the best static mutation rate to use. Our results show that our variable adaptive mutation approach can perform better than static mutation values for large problems. As the problem becomes too large and unmanageable for traditional deterministic methods, our methodology can still produce useful results. When examining larger problems, our results show that having \emph{a priori} knowledge about the best static mutation parameter still does not yield results as good as those generated from our adaptive mutation approach.

\section{Conclusions and Further Study Remarks}
We conclude that using our variable adaptive mutation rate approach yields better results than traditional GA using fixed mutation rates in larger problems. Further research into adjusting the simplified chromosome similarity parameter that replaces mutation rate may yield even better results. Fixed values were used for our chromosome similarity parameter, and varying this parameter may yield more information about how variable adaptive mutation using our approach can be tuned further towards different problem landscapes.

Exploring our variable mutation rate approach in other NP-Complete combinatorial and optimization problems such as the Traveling Salesman Problem or Constraint Satisfaction Problem may show that our approach is valid for a multitude of problems. We conjecture that this is likely since other adaptive mutation methodologies have yielded performance improvements in various problem domains.

Conducting a multivariate analysis and sensitivity analysis of tuning other parameters in conjunction with the variable adaptive mutation rate may yield further performance improvements. For example, variable population size based on the amount of inbreeding may yield different results. This conjecture is based on the fact that with higher mutation rates, survival fitness decreases, resulting in population shrinkage. Coupling the population size with the mutation rate and patterns of fitness in the GA population may yield a new avenue for future study.


% trigger a \newpage just before the given reference
% number - used to balance the columns on the last page
% adjust value as needed - may need to be readjusted if
% the document is modified later
%\IEEEtriggeratref{8}
% The "triggered" command can be changed if desired:
%\IEEEtriggercmd{\enlargethispage{-5in}}

% references section
% NOTE: BibTeX documentation can be easily obtained at:
% http://www.ctan.org/tex-archive/biblio/bibtex/contrib/doc/

% can use a bibliography generated by BibTeX as a .bbl file
% standard IEEE bibliography style from:
% http://www.ctan.org/tex-archive/macros/latex/contrib/supported/IEEEtran/testflow/bibtex
%\bibliographystyle{IEEEtran.bst}
% argument is your BibTeX string definitions and bibliography database(s)
%\bibliography{IEEEabrv,../bib/paper}
%
% <OR> manually copy in the resultant .bbl file
% set second argument of \begin to the number of references
% (used to reserve space for the reference number labels box)


\def\V{\rm vol.~}
\def\N{no.~}
\def\pp{pp.~}
\def\Pot{\it Proc. }
\def\IJCNN{\it International Joint Conference on Neural Networks\rm }
\def\ACC{\it American Control Conference\rm }
\def\SMC{\it IEEE Trans. Systems\rm , \it Man\rm , and \it Cybernetics\rm }

\def\handb{ \it Handbook of Intelligent Control: Neural\rm , \it
    Fuzzy\rm , \it and Adaptive Approaches \rm }

\begin{thebibliography}{22}
\bibitem{cit:1} A. Krogh, J. Hertz and R.G. Palmer,
        {\it Introduction to the Theory of Neural Computation.}
        Addison-Wesley, Redwook City, CA, 1991.

\bibitem{cit:2} D. Marr and T. Poggio, ``Cooperative computation of stereo disparity,"
        {\it Science}, vol. 195, pp. 283-328, 1977.

\bibitem{cit:3} A.L. Yuille and D. Geiger,
        {\it The Handbook of Brain Theory and Neural Networks.}
        Chapter Winner-Take-All Networks, pp. 1228-1231, The MIT Press, 2 edition, 2002.

\bibitem{cit:4} W. Maass, ``Neural computation with winner-take-all as the only nonlinear
operation,"
        {\it Advances in Neural Information Processing Systems}, vol. 12, 2000.

\bibitem{cit:5} W. Maass, ``On the computational power of
winner-take-all,"
        {\it Neural Computation}, vol. 12, pp. 2519-2535, 2000.

\bibitem{cit:6} W.J. Wolfe, D. Mathis, C. Anderson, J. Rothman, M. Gottler, G. Brady, R. Walker, G. Duane and G. Alaghband, ``K-winner networks,"
        {\it IEEE Transactions on Neural Networks}, vol. 2, pp. 310-315, Mar. 1991.

\bibitem{cit:7} J. Wang, ``Analogue winner-take-all neural networks for determining maximum and minimum signals,"
        {\it Int. J. Electronics}, vol. 77, no. 3, pp. 355-367, 1994.

\bibitem{cit:8} K. Urahama and T. Nagao, ``K-winners-take-all ciucuit with O(N) complexity,"
        {\it IEEE Transactions on Neural Networks}, vol. 6, pp. 776-778, May. 1995.

\bibitem{cit:9} B. Sekerkiran and U. Cilingiroglu, ``A CMOS k-winners-take-all circuit with O(N) complexity,"
        {\it IEEE Transactions on Circuits and Systems}, vol. 46, no. 1, Jan. 1999.

\bibitem{cit:10} Jayadeva and S.A. Rahman, ``A neural network with O(N) neurons for ranking N numbers in O(1/N) times,"
        {\it IEEE Transactions on Circuits and Systems I}, vol. 51, no. 10, Oct. 2004.

\bibitem{cit:11} J. Yen, J. Guo and H. Chen, ``A new k-winners-take-all neural network and its array architecture,"
        {\it IEEE Transactions on Neural networks}, vol. 9, no. 5. pp. 901-912, Sept. 1998.

\bibitem{cit:12} B.D. Calvert and C.A. Marinov, ``Another k-winners-take-all analog neural network,"
        {\it IEEE Transactions on Neural Networks}, vol. 11, no. 4, pp. 829-838, Jul. 2000.

\bibitem{cit:13} C.A. Marinov and B.D. Calvert, ``Performance analysis for a k-winners-take-all analog neural network: basic theory,"
        {\it IEEE Transactions on Neural Networks}, vol. 14, no. 4, pp. 766-780, Jul. 2003.

\bibitem{cit:29} C.A. Marinov and J.J. Hopfield, ``Stable computational dynamics for a class of circuits with O(N) interconnections capable of KWTA and rand extractions,"
        {\it IEEE Transactions on Circuits and Systems}, vol. 52, no. 5, pp. 949-959, May 2005.

\bibitem{cit:14} D.W. Tank and J.J. Hopfield, ``Simple neural optimization networks: An A/D converter, signal decision circuit, and a linear programming circuit,"
        {\it IEEE Transactions on Circuits and Systems}, vol. 33, no. 5, pp. 533-541, 1986.

\bibitem{cit:15} J.J. Hopfield and D.W. Tank, ``Computing with neural circuits: A model,"
        {\it Science}, vol. 233, pp. 625-633, 1986.



\bibitem{cit:30} S.B. Liu and J. Wang, ``A simplified dual neural network for quadratic programming with its KWTA application,"
        {\it IEEE Transactions on Neural Networks}, vol. 17, no. 6, pp. 1500-1510, Nov 2006.



\bibitem{cit:16} M.P. Kennedy and L.O. Chua, ``Neural networks for nonlinear programming,"
        {\it IEEE Transactions on Circuits and Systems}, vol. 35, no. 5, pp. 554-562, May. 1988.

\bibitem{cit:17} S. Zhang and A.G. Constantinides, ``Lagrange programming neural networks,"
        {\it IEEE Transactions on Circuits and Systems II}, vol. 39, no. 7, pp. 441-452, Jul. 1992.

\bibitem{cit:18} J. Wang, ``A deterministic annealing neural network for convex programming,"
        {\it Neural Networks}, vol. 5, no. 4, pp. 962-971, 1994.

\bibitem{cit:19} Y. Xia, ``A new neural network for solving linear and quadratic programming problems,"
        {\it IEEE Transactions on Neural Networks}, vol. 7, no. 6, pp. 1544-1547, Nov. 1996.

\bibitem{cit:20} Y. Xia, G. Feng and J. Wang, ``A primal-dual neural network for online resolving constrained kinematic redundancy in robot motion control,"
        {\it IEEE Transactions on Systems, Man and Cybernetics}, vol. 35, no. 1, pp.54-64, Feb. 2005.

\bibitem{cit:21} Y. Xia and J. Wang, ``A dual neural network for kinematic control of redundant robot manipulators,"
        {\it IEEE Transactions on Systems, Man and Cybernetics}, vol. 31, no. 1, pp.147-154, Feb. 2001.

\bibitem{cit:22} Y. Zhang and J. Wang, ``A dual neural network for convex quadratic programming subject to linear equality and inequality constraints,"
        {\it Physics Letters A}, pp. 271-278, Jun. 2002.

\bibitem{cit:23} S. Boyd and L. Vandenbeghe,
        {\it Convex Optimization.}
        Cambridge University Press, Cambridge, UK. 2004.

\bibitem{cit:24} Y. Xia and J. Wang, ``Neural network for solving linear programming problems with bounded variables,"
        {\it IEEE Transactions on Neural Networks}, vol. 6, no. 2, pp. 515-519, 1995.

\bibitem{cit:25} Y. Xia, ``A new neural network for solving linear programming problems and its application,"
        {\it IEEE Transactions on Neural Networks}, vol. 7, no. 2, pp. 525-529, 1996.

\bibitem{cit:26} Y. Xia and J. Wang, ``A recurrent neural network for solving nonlinear convex programs subject to linear constraints,"
        {\it IEEE Transactions on Neural Networks}, vol. 16, no. 2, pp. 378-386, 2005.

\bibitem{cit:27} Y. Xia and J. Wang, ``A general projection neural network for solving monotone variational inequalities and related optimization problems,"
        {\it IEEE Transactions on Neural Networks}, vol. 15, no. 2, pp. 318-328, 2004.



\bibitem{cit:28} D.P. Bertsekas and J.N. Tsitsiklis,
        {\it Parallel and Distributed Computation: Numerical Methods.}
        Prentice-Hall, Englewood Cliffs, NJ, 1989.



\end{thebibliography}

% that's all folks
\end{document}
